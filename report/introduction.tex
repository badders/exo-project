%!TEX root = project.tex
\subsection{Motivations}

Beyond being interesting in the their own right, there are many reasons to look for, and at the properties of exoplanets. Firstly, up until the discovery of the first extra solar planetary system, PSR 1257+ 12 \cite{wolszczan1992planetary}, we were only sure of the existence of planetary bodies in our own solar system. Now we know that our system is not unusual, and it is in fact very common for stars to have many planets of their own \cite{mcarthur2004detection}, and even binary and ternary star systems to have planetary bodies \cite{marcy2002planet}.

Due to the time scales that astronomical phenomena occur on, looking at events such as star and planet formation is impossible over the lifetimes of humans, so instead we have to look at many examples of bodies at varies stages of their life cycles. By looking at planets in other solar systems we can see how they behave across different stages of their evolution, and this can be used to verify our current thinking and models for how solar systems form.

Each planet in our own system has unique and interesting properties too, so by looking elsewhere we can see how common certain characteristics are across the universe, as well as potentially discovering planets with properties completely unlike any in our own solar system.

The favourite motivator is of course the search for life beyond our own. While there is the possibility of some form of life having existed in the past on Mars \cite{mckay1996search} and the possibility of life on some of the outer moons \cite{mckay2005possibilities}, the potential of finding any kind of life, and especially multi-cellular life and even potentially intelligent life is a huge incentive. As techniques for discovering these planets have improved, there have also been observations of Earth sized planets in the so called habitable zone \cite{wordsworth2011gliese}.

\subsection{Methods of Discovery}

\subsubsection{Transit Method}

By observing the flux from a star over period of time, any orbiting planet that passes in front of the star will cause a slight dip in the observed flux. This change in the light curve allows one to calculate the relative sizes of the planet and the star.


Advantages:
\begin{itemize}
    \item Easy to Perform
    \item Can be done at optical wavelengths
\end{itemize}

Disadvantages:
\begin{itemize}
    \item Only works if orbital plane is in line with observer
    \item High false detection rate (REFERENCE)
    \item Poor results for red giant stars (REFERENCE)
\end{itemize}

\subsubsection{Radial Velocity}

\subsubsection{Pulsar timing}

\subsubsection{Gravitational Microlensing}

\subsubsection{Direct Observation}

Model using formulae from \cite{mandel2002analytic}
