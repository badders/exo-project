%!TEX root = project.tex
\subsection{Motivations}

Beyond being interesting in the their own right, there are many reasons to look for, and at the properties of exoplanets. Firstly, up until the discovery of the first extra solar planetary system, PSR 1257+ 12 \cite{wolszczan1992planetary}, we were only sure of the existence of planetary bodies in our own solar system. Now we know that our system is not unusual, and it is in fact very common for stars to have many planets of their own \cite{mcarthur2004detection}, and even binary and ternary star systems to have planetary bodies \cite{marcy2002planet}.

Due to the time scales that astronomical phenomena occur on, looking at events such as star and planet formation is impossible over the lifetimes of humans, so instead we have to look at many examples of bodies at varies stages of their life cycles. By looking at planets in other solar systems we can see how they behave across different stages of their evolution, and this can be used to verify our current thinking and models for how solar systems form.

Each planet in our own system has unique and interesting properties too, so by looking elsewhere we can see how common certain characteristics are across the universe, as well as potentially discovering planets with properties completely unlike any in our own solar system.

The favourite motivator is of course the search for life beyond our own. While there is the possibility of some form of life having existed in the past on Mars \cite{mckay1996search} and the possibility of life on some of the outer moons \cite{mckay2005possibilities}, the potential of finding any kind of life, and especially multi-cellular life and even potentially intelligent life is a huge incentive. As techniques for discovering these planets have improved, there have also been observations of Earth sized planets in the so called habitable zone \cite{wordsworth2011gliese}, making the discovery of distant life a much greater possibility.

\subsection{Methods of Discovery}

\subsubsection{Transit Method}

By observing the flux from a star over period of time, any orbiting planet that passes in front of the star will cause a slight dip in the observed flux. This change in the light curve allows one to calculate the relative sizes of the planet and the star.

\paragraph{Advantages:}
\begin{itemize}
    \item Easy to Perform
    \item Can be done at optical wavelengths
\end{itemize}

\paragraph{Disadvantages:}
\begin{itemize}
    \item Only works if orbital plane is in line with observer
    \item High false detection rate \cite{santerne2012sophie}
    \item Poor results for red giant stars due to their variable surface brightness
\end{itemize}

As well as determining properties of the planet from the transit, by examining the regularity of these transits it is posible to determine information about other planets in the system. This is know as the transit timing variation method.

\subsubsection{Radial Velocity}

Normally we talk about planets orbiting around a star, as this model approximates what we see very well. In actual fact however, the planet and star both orbit their common center of mass. Generally this center of mass is located very close to the center of the star so the, planet orbit model works well. However the star will move slightly, and is especially noticeable for very large planets orbiting closely, so called Hot Jupiters. As the star is moving in its own orbit, small doppler shifts can be seen in the spectrum of the star. Modern spectrometers are so accurate, that even small velocities of the order of $1\units{ms^{-1}}$ can be observed \cite{ge2002externally}.

\paragraph{Advantages:}
\begin{itemize}
    \item Works well for low mass stars
    \item Distance independent
    \item Works for larger inclination range than transit method
\end{itemize}

\paragraph{Disadvantages:}
\begin{itemize}
    \item Works best for large planets
    \item Requires high signal to noise ratio of data
    \item Requires spectrometers rather than just CCD cameras
    \item Poor results for stars with fast rotational velocities
\end{itemize}

\subsubsection{Pulsar timing}

Pulsars rotate very rapidly, and emit radio waves as they do. The regularity of these emission is so precise that even tiny variations can be used to track the motion of the neutron star. In fact the first ever extra solar planet was confirmed using exactly this method in 1992 \cite{wolszczan1992planetary}. While this method is great for finding planets, these systems are quite rare.

\subsubsection{Direct Observation}

Under the right conditions, it can be possible to now see exoplanets directly \cite{lafreniere2010directly}\cite{kuzuhara2013direct}\cite{delorme2013direct}. Obviously stars are very bright compared to their orbiting planets making this technique a bad choice for discovering earth like planets. However large planets, many times the mass of Jupiter are observable. Using a coronagraph on the telescope limits the light from the star \cite{kuchner2002coronagraph}, and observing at infra red wavelengths can overcome the relative brightness problem \cite{delorme2013direct}.
